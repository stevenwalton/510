\section{Project Description}
In this project we create a Q-Learning algorithm that is able to play classic,
or ``retro", videogames. These games are widely considered to be NP-Hard or
many times PSPACE-Hard \cite{DBLP}. PSPACE-Hard problems are not only hard to
compute, but also hard to verify. This should match intuition of many videogames
as it is almost always unclear what an optimal strategy is. This is largely due
to the number of states in a single game. While actions may be few in number,
generally limited to the buttons one can press, each frame in the game can 
generally be considered a new state. This makes even calculating the number
of states extremely difficult. 

At the beginning of this project we set out with the goal to create a Q-learner
that would be able to play Super Mario Kart (SMK) \cite{SMK} through use of 
OpenAI's Retro environment \cite{retro}. Retro creates an environment that 
abstracts out emulation and helps developers focus on creating algorithms,
specifying actions, and rewards. Stretch goals were to get a deep Q network 
(DQN) working and have several agents play against one another.

While we were not able to reach the goal of getting a reinforcement algorithm
playing SMK, we were able to demonstrate feasibility in retro's environment on
several other games, which had already be correctly integrated.

Documentation for the code is included at the end of this document in 
section~\ref{sec:doc}.

