\section{Appendix}
\subsection{SMK Rewards}
\label{sec:SMKRewards}
First off, we need to have a clear goal. Since Super Mario Kart is a competitive
racing game, each player is given a ranking when they finish the race. This
gives us a well defined goal: be first. Since we will be training over a cup,
multiple tracks, we do not need to place first in every race to win the cup, but
placing first in every race will always result in placing first in the cup. The
position can be seen in the bottom right of the screen, in our picture we start
in 8th place (starting position changes depending on finishing position of the 
previous race).

The second factor that is key is that we have a clock, seen in the upper right
of the screen. While we want to place first every time, setting a reward function
purely based on position will only train the agent to beat the game's AI. Instead
we want the agent to finish the race as fast as possible. 

There are other factors to include that we will not consider until combinations 
of the first two are tested. Such factors include that the number of coins, 
with maximum of 10, determine how fast one can go. So having more coins helps
optimize the first two, but may be harder to incorporate. Additionally, there
are pieces on the may that boost the player. We hope that the agent will 
recognize these features on its own and that we do not need to assign a reward
for using these. Including features like these have a higher potential to create
an undesirable reward function. So first we will keep the reward system simple 
and iterate as needed.

\subsection{Determining Finish}
\label{sec:endstate}
As mentioned previously, by incorporating a game that is not already defined by
retro we need to tell retro when we are done. To accomplish this we will actually
need to play this game in an emulator and record the state when finishing a race
as well as when we finish a cup. This task is not yet accomplished but is not 
expected to be exceedingly difficult. 

\subsection{Seeing the Environment}
\label{sec:env}
We have mentioned that we will only be using the top half of the screen, the
reason being is that this is where the agent gains the most information about
their environment. Additionally, when playing in 2 player mode, the bottom half
will be used for that agent's view. Therefore to train properly we need to only
rely on the top half. 

Retro does not create a way for the agent to ``see" its current state. To 
accomplish this we need to turn this Q learner into a Deep Q learner. To do that
we first create a Convolutional Neural Network (CNN) that will determine our
state. Using this we will have a simple network which will then determine the
action that needs to be taken at any given state, position on the track. 

A simple way to think about this flow is with the following image.
\begin{figure}[h]
    \centering
    \includegraphics[width=0.8\textwidth]{conv_agent.png}
    \caption{Example flow of network with Mario Brothers}
\end{figure}
Using the position of Mario and other objects on the screen, the agent needs
to determine if it will run or jump. Our action set will be slightly more
complicated. 

\FloatBarrier
\subsection{RAM Addresses}
\label{sec:RAMVals}
\begin{table}
\centering
    \begin{tabular}{|c|c|}
        \hline
        Address  & Attribute\\\hline
        0x000101 & Milliseconds\\\hline
        0x000102 & Seconds\\\hline
        0x000104 & Minutes\\\hline
        0x000148 & Lap Size\\\hline
        0x000E00 & Coins\\\hline
        0x001040 & Rank\\\hline
        0x0010C1 & Lap\\\hline
        0x0010DC & Checkpoint\\\hline
\end{tabular}
    \caption{RAM Addresses for key features in SMK}
\end{table}
\begin{itemize}
    \item Lap: a value of 133 represents the end of the race
and that one needs to subtract 128 from this number to determine the current
lap.
    \item Rank: Divide by 2 and add 1
    \item Lap Size: Represents the length of the track
    \item Checkpoint: Shows distance from the start of the track. Increments up to the lap size - 1. Eg: if Lap Size = 36, then Checkpoint=$N\%36$.
\end{itemize}


\FloatBarrier
\subsection{Figures}
\begin{figure}[ht]
\centering
\includegraphics[width=0.9\textwidth]{Ram.png}
\caption{Finding Clock Addresses in BizHawk}
\label{fig:hex}
\end{figure}
