\documentclass[12pt,letter]{article}
\usepackage{geometry}\geometry{top=0.75in}
\usepackage{amsmath}
\usepackage{amssymb}
\usepackage{mathtools}
\usepackage{xcolor}	% Color words
%\usepackage{cancel}	% Crossing parts of equations out
\usepackage{tikz}    	% Drawing 
\usepackage{pgfplots}   % Other plotting
\usepgfplotslibrary{colormaps,fillbetween}
\usepackage{placeins}   % Float barrier
\usepackage{hyperref}   % Links
\usepackage{tikz-qtree} % Trees
\usepackage{graphicx}
\usepackage{subcaption}
\usepackage{multicol}
\usepackage{listings}

%\tikzset{
%    treenode/.style = {shape=rectangle, rounded corners, draw, align=center}
%    root/.style     = {treenode, font=\Large}
%    env/.style      = {treenode, font=\ttyfamily\normalsize},
%    dummy/.style    = {circle,draw}
%}

%\tikzset{every tree node/.style={circle,align=center, anchor=west, grow=right}
%	}
\tikzset{every tree node/.style={align=center,minimum width=2em},%, draw},%, circle},
	 grow=right,
	 level distance=1.75cm}

% Don't indent
\setlength{\parindent}{0pt}
% Function to replace \section with a problem name specifically formatted
\newcommand{\problem}[1]{\vspace{3mm}\Large\textbf{{Problem {#1}\vspace{3mm}}}\normalsize\\}
% Formatting function, like \problem
\newcommand{\ppart}[1]{\vspace{2mm}\large\textbf{\\Part {#1})\vspace{2mm}}\normalsize\\}
\newcommand{\documentation}[1]{\vspace{2mm}\large\textbf{\\Documentation{#1}\vspace{2mm}}\normalsize\\}
% Formatting 
\newcommand{\condition}[1]{\vspace{1mm}\textbf{{#1}:}\normalsize\\}

\begin{document}
\title{CIS 510 Assignment 2}
\author{Steven Walton}
\maketitle
\problem{P1}
Program expects an input where the first line is the number of players
and each subsequent line is the grand coalition of a subset of players surrounded
by curly braces and comma separated with the value of that coalition.
An example input file is shown here
\begin{figure}[h]
    \centering
\begin{lstlisting}
4
{1},30
{2},40
{3},25
{4},45
{1,2},50
{1,3},60
{1,4},80
{2,3},55
{2,4},70
{3,4},80
{1,2,3},90
{1,2,4},120
{1,3,4},100
{2,3,4},115
{1,2,3,4},140
\end{lstlisting}
\end{figure}

The program can be run with the following options. Not that only the input file
is required, but it is suggested to specify the output file. 
\begin{figure}[h!]
    \centering
\begin{lstlisting}
python coalition.py --help
usage: coalition.py -i <input file> -o <output file>
-h, --help       prints this message
-i, --input      sets the input file
-o, --output     sets the output file: defaults to optimalCS.txt
\end{lstlisting}
\end{figure}
\FloatBarrier

\problem{Q1}
Consider the ``cross-out" game. In this game one writes down "1,2,3". Player
1 can cross out a single number or any 2 adjacent number (12,23). Player
2 then gets to make the same type of action. The winner is the one who crosses
out the last number.

\ppart{Q1.1}


%\begin{figure}[h]
%    \centering
%    \begin{tikzpicture}[grow=down]
%    \Tree[.P1 
%            [.1 2 3 ] 
%            [.2 1 3 ] 
%            [.3 1 2 ] 
%         ]
%\end{tikzpicture}
%\end{figure}

\ppart{Q1.2}
This is a solved game where player 1 can always win.
\\

To determine this let's first consider a simpler game where a player can only
cross out a single number.
This game is also solved, if it is an odd sized game then player 1 can 
always win, otherwise player 2 can always win. This is because if the size 
is odd then player 1 can place a binary partition and split the game into any 
two sub games.  Player 2 will then be the first player on one sub game 
and player 1 will be the second player in the second sub game. If the second 
player wins any even game, then we can see that player 1 will always win this 
sub game (because it is even and they are the second player for an even 
sized game). 
\\

Now let's consider where a player can cross out any number of adjacent numbers.
Firstly there is the trivial case where player 1 can cross out all the numbers
(I'm assuming this wasn't meant to be possible).
Similarly Player 1 can cross out all but two non-adjacent numbers, then this
is an even game where a player may only cross out a single number (we have 
reduced the game to the single cross out game discussed above). Thinking about
this from subgames, we can see that Player 1 can always reduce any subgame into
the single cross out game. By doing this Player 1 can always put themselves
to be in the first or second position of the new subgames.

\ppart{Q2.1}
Realization Plan Player 1
\begin{align*}
    r_1(\oslash) &= r_1(L) + r_1(R) \\
    r_1(L) &= r_1(Ll) + r_1(Lr)\\
    r_1(R) &= r_1(Rl) + r_1(Rl)\\
    r_1(Ll) &= r_1(LlU) + r_1(LlD)\\
    r_1(Lr) &= r_1(LrU) + r_1(LrD)\\
    r_1(Rl) &= r_1(RlU) + r_1(RlD)\\
    r_1(Rr) &= r_1(RrU) + r_1(RrD)\\
    r_1(\oslash),r_1(L),r_1(R),&r_1(Ll),r_1(Lr),r_1(Rl),r_1(Rr) \geq 0
\end{align*}

Realization Plan Player 2
\begin{align*}
    r_2(\oslash) &= r_2(A) + r_2(B)\\
    r_2(A) &= r_2(AC) + r_2(AD)\\
    r_2(B) &= r_2(BC) + r_2(BD)\\
\end{align*}


\end{document}
