\documentclass[12pt,letter]{article}
\usepackage{geometry}\geometry{top=0.75in}
\usepackage{amsmath}
\usepackage{amssymb}
\usepackage{mathtools}
\usepackage{xcolor}	% Color words
\usepackage{cancel}	% Crossing parts of equations out
\usepackage{tikz}    	% Drawing 
\usepackage{pgfplots}   % Other plotting
\usepgfplotslibrary{colormaps,fillbetween}
\usepackage{placeins}   % Float barrier
\usepackage{hyperref}   % Links
\usepackage{tikz-qtree} % Trees
\usepackage{graphicx}
\usepackage{subcaption}
\usepackage{multicol}

%\tikzset{
%    treenode/.style = {shape=rectangle, rounded corners, draw, align=center}
%    root/.style     = {treenode, font=\Large}
%    env/.style      = {treenode, font=\ttyfamily\normalsize},
%    dummy/.style    = {circle,draw}
%}

%\tikzset{every tree node/.style={circle,align=center, anchor=west, grow=right}
%	}
\tikzset{every tree node/.style={align=center,minimum width=2em},%, draw},%, circle},
	 grow=right,
	 level distance=1.75cm}

% Don't indent
\setlength{\parindent}{0pt}
% Function to replace \section with a problem name specifically formatted
\newcommand{\problem}[1]{\vspace{3mm}\Large\textbf{{Problem {#1}\vspace{3mm}}}\normalsize\\}
% Formatting function, like \problem
\newcommand{\ppart}[1]{\vspace{2mm}\large\textbf{\\Part {#1})\vspace{2mm}}\normalsize\\}
\newcommand{\documentation}[1]{\vspace{2mm}\large\textbf{\\Documentation{#1}\vspace{2mm}}\normalsize\\}
% Formatting 
\newcommand{\condition}[1]{\vspace{1mm}\textbf{{#1}:}\normalsize\\}

\begin{document}
\title{CIS 510 Assignment 1}
\author{Steven Walton}
\maketitle
\problem{1}
Implement the MILP to compute an SSE using Cplex.
\\
Instrcution: the input of your program is two CSV files: \textit{param.csv} and
\textit{payoff.csv}. The format of the \textit{param.csv} file is 
\textit{num\_of\_targets, number\_of\_defender\_resources}. In the \textit{payoff.csv} file
each line consists of five numbers: \textit{target\_id, def\_payoff\_cov, 
def\_payoff\_uncover,att\_payoff\_cov, att\_payoff\_unciver}. The \textbf{output}
of your program is a CSV file named \textit{SSE.csv}. Each line of the output file 
is in the format of \textit{target\_id, def\_coverage\_probability}. A sample
of the three files are provided. 
\\
Submission must include: (i) source codes; (ii) documentary including description
of your program and instructions to run it. Your program will be tested based on
different games.

\documentation{}
Note: This code uses Cplex, which does not work on python 3.7. This code was
written with python 3.6.5 and will be assumed that the user is using a similarly 
compatible python version.

\problem{2}
\ppart{1}
Consider a security game with four targets. The payoffs are given in the following 
table. In each cell, the first number is the defender's payoff and the second is
the attacker's.
\begin{figure*}[h]
\centering
\begin{tabular}{|c|c|c|c|c|}
    \hline
    & t1 & t2 & t3 & t4\\
    \hline
    covered & (1,0) & (3,0) & (8,0) & (8,-1)\\
    \hline
    uncovered & (-1,1) & (0,2) & (0,4) & (-4,4)\\
    \hline
\end{tabular}
\end{figure*}

For a single resource we can see that there are two options that maximize 
the defender's utility: ${t3,t4}$. 

If we have one resource then the first step is to solve the following
\[
    4(1-x) = 4(1-y) - y = 2
\]
We can trivially see that the solution is
\[
    x = \frac12
\]\[
    y = \frac23
\]
At this point our total resources used is $\frac76$ leaving us with $\frac56$ 
resources.

We then solve the following three equations, but we cannot have a distribution 
where the sum is greater than $\frac56$
\[
    4(1-x) = 4(1-y) - y = 2(1-z) = 1
\]
Solving we get
\[
    x = \frac34
\]\[
    y = 1
\]\[
    z = \frac12
\]
Unfortunately this does not work! So we need to do the following
\begin{figure*}[h]
\centering
\begin{tabular}{p{0.5\textwidth}p{0.5\textwidth}}
    {\begin{align*}
    4(1-x) &= 4(1-y) - y\\
    4 - 4x &= 4 - 3y\\
        4x &= 3y \\
        y &= \frac43x
    \end{align*}}
    & 
    {\begin{align*}
        4(1-x) &= 2(1-z)\\
        4 - 4x &= 2 - 2z\\
        -4x +2  &= -2z\\
        z &= 2x - 1
    \end{align*}}
\end{tabular}
\end{figure*}


\end{document}
