\documentclass[12pt,letter]{article}
\usepackage{geometry}\geometry{top=0.75in}
\usepackage{amsmath}
\usepackage{amssymb}
\usepackage{mathtools}
\usepackage{xcolor}	% Color words
\usepackage{cancel}	% Crossing parts of equations out
\usepackage{tikz}    	% Drawing 
\usepackage{pgfplots}   % Other plotting
\usepgfplotslibrary{colormaps,fillbetween}
\usepackage{placeins}   % Float barrier
\usepackage{hyperref}   % Links
\usepackage{tikz-qtree} % Trees
\usepackage{graphicx}
\usepackage{subcaption}
\usepackage{multicol}

%\tikzset{
%    treenode/.style = {shape=rectangle, rounded corners, draw, align=center}
%    root/.style     = {treenode, font=\Large}
%    env/.style      = {treenode, font=\ttyfamily\normalsize},
%    dummy/.style    = {circle,draw}
%}

%\tikzset{every tree node/.style={circle,align=center, anchor=west, grow=right}
%	}
\tikzset{every tree node/.style={align=center,minimum width=2em},%, draw},%, circle},
	 grow=right,
	 level distance=1.75cm}

% Don't indent
\setlength{\parindent}{0pt}
% Function to replace \section with a problem name specifically formatted
\newcommand{\problem}[1]{\vspace{3mm}\Large\textbf{{Problem {#1}\vspace{3mm}}}\normalsize\\}
% Formatting function, like \problem
\newcommand{\ppart}[1]{\vspace{2mm}\large\textbf{\\Part {#1})\vspace{2mm}}\normalsize\\}
\newcommand{\documentation}[1]{\vspace{2mm}\large\textbf{\\Documentation{#1}\vspace{2mm}}\normalsize\\}
% Formatting 
\newcommand{\condition}[1]{\vspace{1mm}\textbf{{#1}:}\normalsize\\}

\begin{document}
\title{CIS 510 Assignment 1}
\author{Steven Walton}
\maketitle
\problem{1}
Implement the MILP to compute an SSE using Cplex.
\\
Instrcution: the input of your program is two CSV files: \textit{param.csv} and
\textit{payoff.csv}. The format of the \textit{param.csv} file is 
\textit{num\_of\_targets, number\_of\_defender\_resources}. In the \textit{payoff.csv} file
each line consists of five numbers: \textit{target\_id, def\_payoff\_cov, 
def\_payoff\_uncover,att\_payoff\_cov, att\_payoff\_unciver}. The \textbf{output}
of your program is a CSV file named \textit{SSE.csv}. Each line of the output file 
is in the format of \textit{target\_id, def\_coverage\_probability}. A sample
of the three files are provided. 
\\
Submission must include: (i) source codes; (ii) documentary including description
of your program and instructions to run it. Your program will be tested based on
different games.

\documentation{}
Note: This code uses Cplex, which does not work on python 3.7. This code was
written with python 3.6.5 and will be assumed that the user is using a similarly 
compatible python version.

\problem{2}
\ppart{1}
Consider a security game with four targets. The payoffs are given in the following 
table. In each cell, the first number is the defender's payoff and the second is
the attacker's.
\begin{figure*}[h]
\centering
\begin{tabular}{|c|c|c|c|c|}
    \hline
    & t1 & t2 & t3 & t4\\
    \hline
    covered & (1,0) & (3,0) & (8,0) & (8,-1)\\
    \hline
    uncovered & (-1,1) & (0,2) & (0,4) & (-4,4)\\
    \hline
    variable  & w & z & y & x\\
    \hline
\end{tabular}
\end{figure*}

For a single resource we can see that there are two options that maximize 
the defender's utility: ${t3,t4}$. 
\\
\large\textbf{One Resource}\normalsize
\\

If we have one resource then the first step is to solve the following
\[
    4(1-x) - x = 4(1-y) = 2
\]
We can trivially see that the solution is
\[
    x = \frac25
\]\[
    y = \frac12
\]
At this point our total resources used is $\frac{9}{10}$ leaving us with $\frac{1}{10}$ 
resources.

\[
    4(1-x) -x = 4(1-y) - y = 2(1-z) = 1
\]
Solving we get
\[
    x = \frac35
\]\[
    y = \frac34
\]\[
    z = \frac12
\]
Unfortunately this does not work $\because x+y+z=\frac{37}{20}>1$! So we need to do the following
\begin{figure*}[h]
\centering
\begin{tabular}{p{0.5\textwidth}p{0.5\textwidth}}
    {\begin{align*}
    4(1-x) - x &= 4(1-y) \\
        4 - 5x &= 4 - 4y\\
        5x &= 4y\\
        y &= \frac54x
    \end{align*}}
    & 
    {\begin{align*}
        4(1-x) -x &= 2(1-z)\\
        4 - 5x &= 2 - 2z \\
        2 - 5x &= -2z\\
        z &= \frac52x - 1
    \end{align*}}
\end{tabular}
\end{figure*}

We can now solve the following equation:
\begin{align*}
    \sum c_i &= 1\\
    x + y + z &= 1\\
    x + \frac54x + \frac52x - 1 &= 1\\
    \frac{x}{8}(8 + 10 + 20) &= 2 \\
    x &= \frac{8}{19}
\end{align*}
And thus we have a result of 
\\
\begin{figure*}[h!]
\centering
\begin{tabular}{c c c}
    $x=\frac{8}{19}$ & $y=\frac{10}{19}$ & $z=\frac{1}{19}$
\end{tabular}
\end{figure*}
Which we can see that the sum here is 1.
\\
\large\textbf{Two Resources}\normalsize
\\
We can cheat a little because of the work we did in the previous example. We know
that we only use up one resource once we consider equality across $x,y,z. 
\therefore$ we can start at that point. We already calculated out that for these
to all be equal then we need $\frac{37}{20}$ resources. While this is $> 1$ it
is $<2$ and $\therefore$ we need to do equality across all 4 variables. Leveraging
our previous work we need to write $w$ in terms of $x$.
\begin{align*}
    4(1-x) - x &= 1-w\\
    4 - 5x &= 1 - w\\
    w &= 5x - 3
\end{align*}
We will now sum up everything and set the r.h.s. to 2, our number of resources.
\begin{align*}
    x + \frac54x + \frac52x - 1 + 5x - 3 &= 2\\
    39x &= 24\\
    x &= \frac{24}{39}
\end{align*}
Thus we get the result
\begin{figure*}[h!]
\centering
\begin{tabular}{c c c c}
    $x=\frac{24}{39}$ & $y=\frac{10}{13}$ & $z=\frac{7}{13}$ & $w=\frac{1}{13}$
\end{tabular}
\end{figure*}
Here we can see that the sum is 2 and we are done. 
\\
\large\textbf{Three Resources}\normalsize
\\
Again we don't have to start over. We can start right here
\begin{align*}
    x + \frac54x + \frac52x - 1 + 5x - 3 &= 3\\
    39x &= 28\\
    x &= \frac{28}{39}
\end{align*}
\begin{figure*}[h!]
\centering
\begin{tabular}{c c c c}
    $x=\frac{28}{39}$ & $y=\frac{35}{39}$ & $z=\frac{31}{39}$ & $w=\frac{23}{39}$
\end{tabular}
\end{figure*}
Checking that the sum is 3 we can verify that we are done.

\end{document}
